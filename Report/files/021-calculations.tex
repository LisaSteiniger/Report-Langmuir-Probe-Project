\chapter{Detektion von Fehlmessungen}\label{chap:calculations}
Für zwei der 36 LS in W7-X ist bekannt, dass es in unregelmäßigen Abständen bei einzelnen Messungen zu Fehlmessungen mit unrealistischen Werten für Elektronendichte und -temperatur kommt. Diese werden durch einen Abfall des Widerstands zwischen Sonde und Ground verursacht, der den Bereich der abgedeckten Spannungswerte schmälert. Es werden beispielsweise nur noch \qty{-160}{V} als unteres Limit erreicht, die exponentielle Form der Kurve ist nicht länger gegeben. Dadurch ist die Bestimmung der Plasmaparameter beeinträchtigt und die ermittelten Werte nicht länger realitätsnah. In besonders gravierenden Fällen können auch Kurzschlüsse verursacht werden, die die elektrischen Bauteile im Schaltkreis mit den LS beschädigen können.\\

Die Gründe für diese spontanen, zeitlich begrenzten Widerstandsabfälle sind nicht bekannt. Im Verdacht stehen jedoch kohlenstoffhaltige Ablagerung auf den Keramikfassungen der LS. Die Fassungen sollten eigentlich isolierend sein, durch die Ablagerungen kann allerdings elektrische Leitfähigkeit erreicht werden. Dazu passt auch die gehäufte Beobachtung dieses Phänomens bei erhöhtem Vorkommen von Kohlenstoff-Flakes im Plasma.\\

Es existiert bislang keine Programmroutine, die diese Spannungsabfälle erkennen kann und demnach auch keine Liste mit betroffenen Messungen. Dadurch sind alle Daten dieser beiden LS unzuverlässig und nur unter Vorbehalt zu verwenden. Um dieses Problem zu lösen, muss eine systematische Überprüfung aller Messungen der betroffenen LS erfolgen. Dadurch kann eventuell auch eine Systematik hinter dem Auftreten der Widerstandsabfälle gefunden werden. Das könnte Rückschlüsse auf die Gründe dieses Phänomens sowie Strategien zur Vermeidung desselben ermöglichen.\\

Eine derartige Routine kann einerseits auf der Analyse der erreichten Spannungswerte basieren. Da allerdings durch kurzschlussbedingte Schäden an Vorwiderständen mitunter auch in normalen Messungen ohne Widerstandsabfall keine \qty{-180}{V} mehr erreicht werden, kann dieses Kriterium nur als Indikator dienen. Daher wird ein weiterer Prüfmechanismus etabliert, der den Widerstand des ganzen Schaltkreises testet. Zusätzlich muss der Anschluss der Sonde an den Schaltkreis geprüft werden. Anschließend müssen für betroffene Messungen Betriebs- und Plasmaparameter ausgelesen und verglichen werden, sodass eine mögliche Systematik erkennbar wird. Die genaue Umsetzung dieses Vorgehens wird im Folgenden erklärt.

\section{Detektion von Widerstandsabfällen}
\subsection{Analyse des abgedeckten Spannungsbereichs}
Der abgedeckte Bereich der Spannung liegt im Idealfall bei \qty{-180}{V} bis \qty{20}{V}. Als einfachster Indikator kann daher der Maximal- und Minimalwert von \(V\) für jede Periode des Sinussignals bestimmt und mit diesen Limits verglichen werden (globale Maxima einer Periode). Dabei wird eine Toleranz von \qty{}{\%} angesetzt.\\

Wie bereits erwähnt bedeuten niedrigere Maximal- oder Minimalspannungen jedoch nicht zwangsläufig einen Widerstandsabfall. Ist bei einem vorangegangenen Widerstandsabfall durch die hohe Stromstärke der Vorwiderstand \(R_s\) beschädigt worden, so können danach die normalen Spannungslimits nicht länger erreicht werden. Das ist dann unabhängig von erneuten Widerstandsabfällen. Damit Messungen, die davon betroffen sind, nicht als Fehlmessungen deklariert werden, kann zusätzlich geprüft werden, wie sich die Extremwerte der Spannung entwickeln. Ist in einer Periode noch ein Minimum von \qty{-150}{V} erreicht worden, in der nächsten hingegen nur noch \qty{-125}{V}, so kann von einer Fehlmessung ausgegangen werden. Sind die Werte der Extrema hingegen konstant, so muss eine genauere Untersuchung erfolgen, um eine sichere Einordnung in korrekte oder Fehlmessung vorzunehmen.   

\subsection{Analyse des Widerstands des Schaltkreises}
Eine Möglichkeit, genauer nachzuprüfen, ob ein Widerstandsabfall oder ein beschädigter Vorwiderstand die Ursache eines zu geringen Spannungsbereichs sind, kann der Gesamtwiderstand des Schaltkreises bestimmt werden. Dazu wird die von Widerstandsabfällen unbeeinflusste Inputspannung \(V_{in}\) durch \(I_p\) dividiert. Ausgeschlossen werden von der Analyse solche Wertepaare mit \(\vert V_{in} \vert <\) \qty{0,5}{V}, da diese automatisch zu sehr niedrigen Widerstandswerten führen würden. Anschließend werden immer \num{50} Widerstandswerte gemittelt und gegen den Grenzwert von \qty{300}{\ohm} geprüft. Liegen sie darunter, wird die Messung als Fehlmessung betrachtet.

\section{Bestimmung der Charakteristika der Entladung}
\subsection{Auslesen der Betriebsparameter}
Magnetfeldkonfiguration, experiment description, program abort, cryo pump status, gas inlets
\subsection{Auslesen der Plasmaparameter}
Heizleistung und -quelle, Prad und frad (detachement), HHeRatio, ne, Te, Iplasma, Wdia 