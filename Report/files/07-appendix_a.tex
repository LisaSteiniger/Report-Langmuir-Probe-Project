\chapter{Appendix A: Formelzeichen}\label{chap:appendixA}
\begin{align*}
	_e &- \text{Index für elektronenbezogene Größen}\\
	_i &- \text{Index für ionenbezogene Größen}\\
	_{ero} &- \text{Index für erosionsbezogene Größen}\\
	_{dep} &- \text{Index für deponierungsbezogene Größen}\\\\
	c &- \text{Lichtgeschwindigkeit im Vakuum}\\
	h &- \text{Plancksches Wirkungsquantum}\\
	k_B &- \text{Boltzmann Konstante}\\
	N_A &- \text{Avogadrozahl}\\
	t &- \text{Zeit}\\
	g &- \text{Wahrscheinlichkeitsverteilung der Teilchenenergie}\\
	U &- \text{Energiedichte in Planckschem Strahlungsgesetz,}\nonumber\\
	&\text{ \quad Spannung bei Langmuir Sonden}\nonumber\\
	f &- \text{Frequenz}\\
	\lambda &- \text{Wellenlänge}\\
	I &- \text{Intensität in Bezug auf Licht,}\nonumber\\
	&\text{ \quad Stromstärke bei Langmuir Sonden}\nonumber\\
	l &- \text{Länge der Strikeline}\\
	A &- \text{Fläche des Divertors, die mit dem Plasma in Berührung kommt}\\
	M &- \text{Molare Masse}\\
	Z &- \text{Kernladungszahl}\\
	m &- \text{Masse}\\
	n &- \text{Teilchendichte}\\
	\rho &- \text{Materialdichte}\\
	f_i &- \text{Ionenkonzentration}\\
	q_i	&- \text{Ionisationszustand}\\
	T &- \text{Temperatur}\\
	T_s &- \text{Oberflächentemperatur des Targets}\\
	\Gamma &- \text{Teilchenflussdichte}\\
	\Delta &- \text{Schichtdicke}\\
	Y &- \text{Zerstäubungsausbeute}\\
	Y_{chem} &- \text{Ausbeute durch chemische Erosion}\\
	Y^{damage} &- \text{Teil von \(Y\) für Ionen-induzierte Desorption}\\
	Y^{therm} &- \text{Teil von \(Y\) für Ionen-induzierte Desorption}\\
	Y^{surf} &- \text{\(Y\) für thermisch aktivierte Kohlenwasserstoff-Emission}\\
	Y^{self} &- \text{Selbstzerstäubungsausbeute}\\
	s_n &- \text{Nuklearer Wirkungsquerschnitt (nuclear stopping cross section) für}\nonumber\\
	&\text{ \quad Krypton-Kohlenstoff Potential}\nonumber\\
	s_n^{TF} &- \text{Nuklearer Wirkungsquerschnitt (nuclear stopping cross section)}\nonumber\\
	&\text{ \quad für Thomas-Fermi Potential}\nonumber\\
	E &- \text{Energie des auftreffenden Ions}\\
	E_{TF} &- \text{Thomas-Fermi Energie}\\
	E_{th} &- \text{Schwellenenergie für physikalische Zerstäubung}\\
	E_{thd} &- \text{Schwellenenergie für Ionen-induzierte Desorption}\\
	E_{ths} &- \text{Schwellenenergie für thermisch aktivierte Kohlenwasserstoff-Emission}\\
	E_s &- \text{Sublimationswärme}\\
	\epsilon &- \text{Reduzierte Energie}\\
	a_L &- \text{Lindhard Screening-Länge}\\
	\gamma_k &- \text{Kinematischer Faktor}\\
	\alpha_0 &- \text{Korrekturfaktor}\\
	\alpha &- \text{Einfallswinkel der Ionen auf den Divertor zur der Flächennormalen}\\
	\alpha_{max} &- \text{\(\alpha\) mit maximaler physikalischer Zerstäubungsausbeute}\\
	\zeta &- \text{Einfallswinkel der Magnetfeldlinien auf den glatten Divertor gemessen}\nonumber\\
		&\text{ \quad von der Oberfläche zum Lot}\nonumber\\
	\Phi &- \text{Toroidaler Winkel}\\
	P_{redep} &- \text{Wahrscheinlichkeit der Wiederablagerung}\\
	s &- \text{Haftungskoeffizient}\\
	Q_y &- \text{Fitparameter}\\
	f_y &- \text{Yamamura Parameter}\\
	b, c, f , Y_0 &- \text{Fitparameter für Modell nach \cite{BehrischEckstein}}\\
	c_i, s_i, C_d &- \text{Parameter für chemische Erosion nach \cite{PWI-Dirk}}\\
	C, D, c^{sp3} &- \text{Parameter für chemische Erosion nach \cite{RothChemErosion}}\\
	Z_{eff} &- \text{Absoluter Ladungszustand des Plasmas}
\end{align*}


