\chapter{Detektion von Fehlmessungen}\label{chap:calculations}
Für zwei der 36 LS in W7-X ist bekannt, dass es in unregelmäßigen Abständen bei einzelnen Messungen zu Fehlmessungen mit unrealistischen Werten für Elektronendichte und -temperatur kommt. Diese werden durch einen Abfall des Widerstands zwischen Sonde und Ground verursacht, der den Bereich der abgedeckten Spannungswerte schmälert. Es werden beispielsweise nur noch -160 V als unteres Limit erreicht, die exponentielle Form der Kurve ist nicht länger gegeben. Dadurch ist die Bestimmung der Plasmaparameter beeinträchtigt und die ermittelten Werte nicht länger realitätsnah. In besonders gravierenden Fällen können auch Kurzschlüsse verursacht werden, die die elektrischen Bauteile im Schaltkreis mit den LS beschädigen können.\\

Die Gründe für diese spontanen, zeitlich begrenzten Widerstandsabfälle sind nicht bekannt. Im Verdacht stehen jedoch kohlenstoffhaltige Ablagerung auf den Keramikfassungen der LS. Die Fassungen sollten eigentlich isolierend sein, durch die Ablagerungen kann allerdings elektrische Leitfähigkeit erreicht werden. Dazu passt auch die gehäufte Beobachtung dieses Phänomens bei erhöhtem Vorkommen von Kohlenstoff-Flakes im Plasma.\\

Es existiert bislang keine Programmroutine, die diese Spannungsabfälle erkennen kann und demnach auch keine Liste mit betroffenen Messungen. Dadurch sind alle Daten dieser beiden LS unzuverlässig und nur unter Vorbehalt zu verwenden. Um dieses Problem zu lösen, muss eine systematische Überprüfung aller Messungen der betroffenen LS erfolgen. Dadurch kann eventuell auch eine Systematik hinter dem Auftreten der Widerstandsabfälle gefunden werden. Das könnte Rückschlüsse auf die Gründe dieses Phänomens sowie Strategien zur Vermeidung desselben ermöglichen.\\

Eine derartige Routine würde voraussichtlich auf der Analyse der erreichten Spannungswerte basieren. Außerdem müssen der Anschluss der Sonde an den Schaltkreis geprüft und die Einflüsse durch kurzschlussbedingte Schäden an Vorwiderständen in Betracht gezogen werden. Anschließend müssen für betroffene Messungen Betriebs- und Plasmaparameter ausgelesen und verglichen werden, sodass eine mögliche Systematik erkennbar wird.