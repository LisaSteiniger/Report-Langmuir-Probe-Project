\chapter{Appendix D: Abbildungen}\label{chap:appendixC}

\section{Erosions-/Deponierungsraten der ganzen Kampagne OP2.2 und OP2.3}

\begin{figure}[!htb] 
	\centering
	\includegraphics[scale=0.45]{figures/OP22\_320K\_totalErosionRates\_wholeCampaign\_Iota\_extrapolated.png}
	\caption{Durchschnittliche Erosions-/Deponierungsrate von Kohlenstoff am Divertor in OP2.2 in Abhängigkeit vom Abstand vom Pumpspalt für low-iota und high-iota Bereiche der oberen und unteren Divertoreinheit (DU). Rot und mit positivem Vorzeichen dargestellt ist die Bruttoerosion, blau und mit negativem Vorzeichen die Bruttodeponierung und schwarz die Nettoerosion (neg. Vorzeichen) beziehungsweise -deponierung (pos. Vorzeichen).}\label{fig:rateOp22}
\end{figure}

\begin{figure}[!htb] 
	\centering
	\includegraphics[scale=0.45]{figures/OP23\_320K\_totalErosionRates\_wholeCampaign\_Iota\_extrapolated.png}
	\caption{Durchschnittliche Erosions-/Deponierungsrate von Kohlenstoff am Divertor in OP2.3 in Abhängigkeit vom Abstand vom Pumpspalt für low-iota und high-iota Bereiche der oberen und unteren Divertoreinheit (DU). Rot und mit positivem Vorzeichen dargestellt ist die Bruttoerosion, blau und mit negativem Vorzeichen die Bruttodeponierung und schwarz die Nettoerosion (neg. Vorzeichen) beziehungsweise -deponierung (pos. Vorzeichen).}\label{fig:rateOp23}
\end{figure}

\begin{figure}[!htb] 
	\centering
	\includegraphics[scale=0.45]{figures/OP22\_320K\_compareErosionRates\_wholeCampaignextrapolatedNo4No26.png}
	\caption{Durchschnittliche Erosions-/Deponierungsrate von Kohlenstoff am Divertor in OP2.2 in Abhängigkeit vom Abstand vom Pumpspalt für low-iota und high-iota Bereiche der oberen und unteren Divertoreinheit (DU) mit Grenzwerten bei niedrigerer und höherer Kohlenstoffkonzentration (grau). Nettoerosion liegt bei Werten unter \qty[per-mode=symbol]{0}{\nm\per\s} vor, Nettodeponierung bei Werten darüber.}\label{fig:compRateOp22}
\end{figure}

\begin{figure}[!htb] 
	\centering
	\includegraphics[scale=0.45]{figures/OP23\_320K\_compareErosionRates\_wholeCampaignextrapolatedNo4No26.png}
	\caption{Durchschnittliche Erosions-/Deponierungsrate von Kohlenstoff am Divertor in OP2.3 in Abhängigkeit vom Abstand vom Pumpspalt für low-iota und high-iota Bereiche der oberen und unteren Divertoreinheit (DU) mit Grenzwerten bei niedrigerer und höherer Kohlenstoffkonzentration (grau). Nettoerosion liegt bei Werten unter \qty[per-mode=symbol]{0}{\nm\per\s} vor, Nettodeponierung bei Werten darüber.}\label{fig:compRateOp23}
\end{figure}
\clearpage

\section{Low-Iota Konfiguration DBM}
\begin{figure}[!htb] 
	\centering
	\includegraphics[scale=0.45]{figures/neOP22DBMAverageAllPositionsHighIota.png}
	\caption{Durchschnittliche Elektronendichte für DBM Konfigurationen in OP2.2 in Ab\-hängigkeit vom Abstand vom Pumpspalt für low-iota und high-iota Bereiche der oberen und unteren Divertoreinheit (DU).}
\end{figure}

\begin{figure}[!htb] 
	\centering
	\includegraphics[scale=0.45]{figures/neOP23DBMAverageAllPositionsHighIota.png}
	\caption{Durchschnittliche Elektronendichte für DBM Konfigurationen in OP2.3 in Ab\-hängigkeit vom Abstand vom Pumpspalt für low-iota und high-iota Bereiche der oberen und unteren Divertoreinheit (DU).}
\end{figure}

\begin{figure}[!htb] 
	\centering
	\includegraphics[scale=0.45]{figures/TeOP22DBMAverageAllPositionsHighIota.png}
	\caption{Durchschnittliche Elektronentemperatur für DBM Konfigurationen in OP2.2 in Abhängigkeit vom Abstand vom Pumpspalt für low-iota und high-iota Bereiche der oberen und unteren Divertoreinheit (DU).}
\end{figure}

\begin{figure}[!htb] 
	\centering
	\includegraphics[scale=0.45]{figures/TeOP23DBMAverageAllPositionsHighIota.png}
	\caption{Durchschnittliche Elektronentemperatur für DBM Konfigurationen in OP2.3 in Abhängigkeit vom Abstand vom Pumpspalt für low-iota und high-iota Bereiche der oberen und unteren Divertoreinheit (DU).}
\end{figure}

\begin{figure}[!htb] 
	\centering
	\includegraphics[scale=0.45]{figures/TsOP22DBMAverageAllPositionsHighIota.png}
	\caption{Durchschnittliche Oberflächentemperatur des Divertors für DBM Konfigurationen in OP2.2 in Abhängigkeit vom Abstand vom Pumpspalt für low-iota und high-iota Bereiche der oberen und unteren Divertoreinheit (DU).}
\end{figure}

\begin{figure}[!htb] 
	\centering
	\includegraphics[scale=0.45]{figures/TsOP23DBMAverageAllPositionsHighIota.png}
	\caption{Durchschnittliche Oberflächentemperatur des Divertors für DBM Konfigurationen in OP2.3 in Abhängigkeit vom Abstand vom Pumpspalt für low-iota und high-iota Bereiche der oberen und unteren Divertoreinheit (DU).}
\end{figure}

\begin{figure}[!htb] 
	\centering
	\includegraphics[scale=0.45]{figures/OP22\_320K\_totalErosionLayers\_DBM\_Iota\_extrapolated.png}
	\caption{Erodierte/deponierte Schichtdicke von Kohlenstoff am Divertor für DBM Konfigurationen in OP2.2 in Abhängigkeit vom Abstand vom Pumpspalt für low-iota und high-iota Bereiche der oberen und unteren Divertoreinheit (DU). Rot und mit positivem Vorzeichen dargestellt ist die Bruttoerosion, blau und mit negativem Vorzeichen die Bruttodeponierung und schwarz die Nettoerosion (neg. Vorzeichen) beziehungsweise -deponierung (pos. Vorzeichen).}
\end{figure}

\begin{figure}[!htb] 
	\centering
	\includegraphics[scale=0.45]{figures/OP23\_320K\_totalErosionLayers\_DBM\_Iota\_extrapolated.png}
	\caption{Erodierte/deponierte Schichtdicke von Kohlenstoff am Divertor für DBM Konfigurationen in OP2.3 in Abhängigkeit vom Abstand vom Pumpspalt für low-iota und high-iota Bereiche der oberen und unteren Divertoreinheit (DU). Rot und mit positivem Vorzeichen dargestellt ist die Bruttoerosion, blau und mit negativem Vorzeichen die Bruttodeponierung und schwarz die Nettoerosion (neg. Vorzeichen) beziehungsweise -deponierung (pos. Vorzeichen).}
\end{figure}
\clearpage

\section{Standardkonfiguration EIM}
\begin{figure}[!htb] 
	\centering
	\includegraphics[scale=0.45]{figures/neOP22EIMAverageAllPositionsHighIota.png}
	\caption{Durchschnittliche Elektronendichte für EIM Konfigurationen in OP2.2 in Ab\-hängigkeit vom Abstand vom Pumpspalt für low-iota und high-iota Bereiche der oberen und unteren Divertoreinheit (DU).}
\end{figure}

\begin{figure}[!htb] 
	\centering
	\includegraphics[scale=0.45]{figures/neOP23EIMAverageAllPositionsHighIota.png}
	\caption{Durchschnittliche Elektronendichte für EIM Konfigurationen in OP2.3 in Ab\-hängigkeit vom Abstand vom Pumpspalt für low-iota und high-iota Bereiche der oberen und unteren Divertoreinheit (DU).}
\end{figure}

\begin{figure}[!htb] 
	\centering
	\includegraphics[scale=0.45]{figures/TeOP22EIMAverageAllPositionsHighIota.png}
	\caption{Durchschnittliche Elektronentemperatur für EIM Konfigurationen in OP2.2 in Abhängigkeit vom Abstand vom Pumpspalt für low-iota und high-iota Bereiche der oberen und unteren Divertoreinheit (DU).}
\end{figure}

\begin{figure}[!htb] 
	\centering
	\includegraphics[scale=0.45]{figures/TeOP23EIMAverageAllPositionsHighIota.png}
	\caption{Durchschnittliche Elektronentemperatur für EIM Konfigurationen in OP2.3 in Abhängigkeit vom Abstand vom Pumpspalt für low-iota und high-iota Bereiche der oberen und unteren Divertoreinheit (DU).}
\end{figure}

\begin{figure}[!htb] 
	\centering
	\includegraphics[scale=0.45]{figures/TsOP22EIMAverageAllPositionsHighIota.png}
	\caption{Durchschnittliche Oberflächentemperatur des Divertors für EIM Konfigurationen in OP2.2 in Abhängigkeit vom Abstand vom Pumpspalt für low-iota und high-iota Bereiche der oberen und unteren Divertoreinheit (DU).}
\end{figure}

\begin{figure}[!htb] 
	\centering
	\includegraphics[scale=0.45]{figures/TsOP23EIMAverageAllPositionsHighIota.png}
	\caption{Durchschnittliche Oberflächentemperatur des Divertors für EIM Konfigurationen in OP2.3 in Abhängigkeit vom Abstand vom Pumpspalt für low-iota und high-iota Bereiche der oberen und unteren Divertoreinheit (DU).}
\end{figure}

\begin{figure}[!htb] 
	\centering
	\includegraphics[scale=0.45]{figures/OP22\_320K\_totalErosionLayers\_EIM\_Iota\_extrapolated.png}
	\caption{Erodierte/deponierte Schichtdicke von Kohlenstoff am Divertor für EIM Konfigurationen in OP2.2 in Abhängigkeit vom Abstand vom Pumpspalt für low-iota und high-iota Bereiche der oberen und unteren Divertoreinheit (DU). Rot und mit positivem Vorzeichen dargestellt ist die Bruttoerosion, blau und mit negativem Vorzeichen die Bruttodeponierung und schwarz die Nettoerosion (neg. Vorzeichen) beziehungsweise -deponierung (pos. Vorzeichen).}
\end{figure}

\begin{figure}[!htb] 
	\centering
	\includegraphics[scale=0.45]{figures/OP23\_320K\_totalErosionLayers\_EIM\_Iota\_extrapolated.png}
	\caption{Erodierte/deponierte Schichtdicke von Kohlenstoff am Divertor für EIM Konfigurationen in OP2.3 in Abhängigkeit vom Abstand vom Pumpspalt für low-iota und high-iota Bereiche der oberen und unteren Divertoreinheit (DU). Rot und mit positivem Vorzeichen dargestellt ist die Bruttoerosion, blau und mit negativem Vorzeichen die Bruttodeponierung und schwarz die Nettoerosion (neg. Vorzeichen) beziehungsweise -deponierung (pos. Vorzeichen).}
\end{figure}
\clearpage

\section{High-Mirror Konfiguration KJM}
\begin{figure}[!htb] 
	\centering
	\includegraphics[scale=0.45]{figures/neOP22KJMAverageAllPositionsHighIota.png}
	\caption{Durchschnittliche Elektronendichte für KJM Konfigurationen in OP2.2 in Ab\-hängigkeit vom Abstand vom Pumpspalt für low-iota und high-iota Bereiche der oberen und unteren Divertoreinheit (DU).}
\end{figure}

\begin{figure}[!htb] 
	\centering
	\includegraphics[scale=0.45]{figures/neOP23KJMAverageAllPositionsHighIota.png}
	\caption{Durchschnittliche Elektronendichte für KJM Konfigurationen in OP2.3 in Ab\-hängigkeit vom Abstand vom Pumpspalt für low-iota und high-iota Bereiche der oberen und unteren Divertoreinheit (DU).}
\end{figure}

\begin{figure}[!htb] 
	\centering
	\includegraphics[scale=0.45]{figures/TeOP22KJMAverageAllPositionsHighIota.png}
	\caption{Durchschnittliche Elektronentemperatur für KJM Konfigurationen in OP2.2 in Abhängigkeit vom Abstand vom Pumpspalt für low-iota und high-iota Bereiche der oberen und unteren Divertoreinheit (DU).}
\end{figure}

\begin{figure}[!htb] 
	\centering
	\includegraphics[scale=0.45]{figures/TeOP23KJMAverageAllPositionsHighIota.png}
	\caption{Durchschnittliche Elektronentemperatur für KJM Konfigurationen in OP2.3 in Abhängigkeit vom Abstand vom Pumpspalt für low-iota und high-iota Bereiche der oberen und unteren Divertoreinheit (DU).}
\end{figure}

\begin{figure}[!htb] 
	\centering
	\includegraphics[scale=0.45]{figures/TsOP22KJMAverageAllPositionsHighIota.png}
	\caption{Durchschnittliche Oberflächentemperatur des Divertors für KJM Konfigurationen in OP2.2 in Abhängigkeit vom Abstand vom Pumpspalt für low-iota und high-iota Bereiche der oberen und unteren Divertoreinheit (DU).}
\end{figure}

\begin{figure}[!htb] 
	\centering
	\includegraphics[scale=0.45]{figures/TsOP23KJMAverageAllPositionsHighIota.png}
	\caption{Durchschnittliche Oberflächentemperatur des Divertors für KJM Konfigurationen in OP2.3 in Abhängigkeit vom Abstand vom Pumpspalt für low-iota und high-iota Bereiche der oberen und unteren Divertoreinheit (DU).}
\end{figure}

\begin{figure}[!htb] 
	\centering
	\includegraphics[scale=0.45]{figures/OP22\_320K\_totalErosionLayers\_KJM\_Iota\_extrapolated.png}
	\caption{Erodierte/deponierte Schichtdicke von Kohlenstoff am Divertor für KJM Konfigurationen in OP2.2 in Abhängigkeit vom Abstand vom Pumpspalt für low-iota und high-iota Bereiche der oberen und unteren Divertoreinheit (DU). Rot und mit positivem Vorzeichen dargestellt ist die Bruttoerosion, blau und mit negativem Vorzeichen die Bruttodeponierung und schwarz die Nettoerosion (neg. Vorzeichen) beziehungsweise -deponierung (pos. Vorzeichen).}
\end{figure}

\begin{figure}[!htb] 
	\centering
	\includegraphics[scale=0.45]{figures/OP23\_320K\_totalErosionLayers\_KJM\_Iota\_extrapolated.png}
	\caption{Erodierte/deponierte Schichtdicke von Kohlenstoff am Divertor für KJM Konfigurationen in OP2.3 in Abhängigkeit vom Abstand vom Pumpspalt für low-iota und high-iota Bereiche der oberen und unteren Divertoreinheit (DU). Rot und mit positivem Vorzeichen dargestellt ist die Bruttoerosion, blau und mit negativem Vorzeichen die Bruttodeponierung und schwarz die Nettoerosion (neg. Vorzeichen) beziehungsweise -deponierung (pos. Vorzeichen).}
\end{figure}
\clearpage

\section{High-Iota Konfiguration FTM}
\begin{figure}[!htb] 
	\centering
	\includegraphics[scale=0.45]{figures/neOP22FTMAverageAllPositionsHighIota.png}
	\caption{Durchschnittliche Elektronendichte für FTM Konfigurationen in OP2.2 in Ab\-hängigkeit vom Abstand vom Pumpspalt für low-iota und high-iota Bereiche der oberen und unteren Divertoreinheit (DU).}
\end{figure}

\begin{figure}[!htb] 
	\centering
	\includegraphics[scale=0.45]{figures/neOP23FTMAverageAllPositionsHighIota.png}
	\caption{Durchschnittliche Elektronendichte für FTM Konfigurationen in OP2.3 in Ab\-hängigkeit vom Abstand vom Pumpspalt für low-iota und high-iota Bereiche der oberen und unteren Divertoreinheit (DU).}
\end{figure}

\begin{figure}[!htb] 
	\centering
	\includegraphics[scale=0.45]{figures/TeOP22FTMAverageAllPositionsHighIota.png}
	\caption{Durchschnittliche Elektronentemperatur für FTM Konfigurationen in OP2.2 in Abhängigkeit vom Abstand vom Pumpspalt für low-iota und high-iota Bereiche der oberen und unteren Divertoreinheit (DU).}
\end{figure}

\begin{figure}[!htb] 
	\centering
	\includegraphics[scale=0.45]{figures/TeOP23FTMAverageAllPositionsHighIota.png}
	\caption{Durchschnittliche Elektronentemperatur für FTM Konfigurationen in OP2.3 in Abhängigkeit vom Abstand vom Pumpspalt für low-iota und high-iota Bereiche der oberen und unteren Divertoreinheit (DU).}
\end{figure}

\begin{figure}[!htb] 
	\centering
	\includegraphics[scale=0.45]{figures/TsOP22FTMAverageAllPositionsHighIota.png}
	\caption{Durchschnittliche Oberflächentemperatur des Divertors für FTM Konfigurationen in OP2.2 in Abhängigkeit vom Abstand vom Pumpspalt für low-iota und high-iota Bereiche der oberen und unteren Divertoreinheit (DU).}
\end{figure}

\begin{figure}[!htb] 
	\centering
	\includegraphics[scale=0.45]{figures/TsOP23FTMAverageAllPositionsHighIota.png}
	\caption{Durchschnittliche Oberflächentemperatur des Divertors für FTM Konfigurationen in OP2.3 in Abhängigkeit vom Abstand vom Pumpspalt für low-iota und high-iota Bereiche der oberen und unteren Divertoreinheit (DU).}
\end{figure}

\begin{figure}[!htb] 
	\centering
	\includegraphics[scale=0.45]{figures/OP22\_320K\_totalErosionLayers\_FTM\_Iota\_extrapolated.png}
	\caption{Erodierte/deponierte Schichtdicke von Kohlenstoff am Divertor für FTM Konfigurationen in OP2.2 in Abhängigkeit vom Abstand vom Pumpspalt für low-iota und high-iota Bereiche der oberen und unteren Divertoreinheit (DU). Rot und mit positivem Vorzeichen dargestellt ist die Bruttoerosion, blau und mit negativem Vorzeichen die Bruttodeponierung und schwarz die Nettoerosion (neg. Vorzeichen) beziehungsweise -deponierung (pos. Vorzeichen).}
\end{figure}

\begin{figure}[!htb] 
	\centering
	\includegraphics[scale=0.45]{figures/OP23\_320K\_totalErosionLayers\_FTM\_Iota\_extrapolated.png}
	\caption{Erodierte/deponierte Schichtdicke von Kohlenstoff am Divertor für FTM Konfigurationen in OP2.3 in Abhängigkeit vom Abstand vom Pumpspalt für low-iota und high-iota Bereiche der oberen und unteren Divertoreinheit (DU). Rot und mit positivem Vorzeichen dargestellt ist die Bruttoerosion, blau und mit negativem Vorzeichen die Bruttodeponierung und schwarz die Nettoerosion (neg. Vorzeichen) beziehungsweise -deponierung (pos. Vorzeichen).}
\end{figure}
