\chapter{Messungen und Datenzugriff} \label{chap:dataAccess}
In dieser Bachelorarbeit wurde mit der Elektronendichte, der Elektronentemperatur und der Oberflächentemperatur der Divertortargets gerechnet. Die zugehörigen Datensätze stammen von Langmuir Sonden und Infrarot-Kamerasysteme aus dem Experimentalbetrieb von Wendelstein 7-X in OP2.2 und OP2.3. Diese Diagnostiken, ihre Funktionsweise und Besonderheiten im entstehenden Datenprofil werden in diesem Abschnitt erklärt. Zusätzlich wird auf die Helium-Beam-Diagnostik eingegangen, deren Messdaten zur Validierung der Daten der Langmuir Sonden verwendet wurden. Außerdem werden Spektroskopie-Diagnostiken vorgestellt, deren Daten Rückschlüsse auf die Ionenkonzentrationen im Randplasma erlauben.

\section{Langmuir Sonden}\label{sec:LP}
Langmuir Sonden sind Messgeräte, die im Grunde aus einem einzelnen, nicht isolierten Draht bestehen, welcher ins Plasma gehalten wird \cite{LPfunction, LPBochum}. Sie liefern unter anderem lokale Messdaten für Elektronendichte und Elektronentemperatur. Das dahinterstehende Messprinzip beruht darauf, dass das Einführen der Elektrode ins Plasma zur Wechselwirkung mit eben jenem führt.\\

Elektronen sind aufgrund ihrer niedrigeren Masse mobiler als Ionen, ein Effekt, der durch ihre höhere Temperatur noch verstärkt wird \cite{PWI-Dirk}(S. 109ff). Eine dem Plasmapotential gegenüber neutral geladene Oberfläche würden Elektronen demnach schneller erreichen und negativ aufladen. Die Elektrode einer Langmuir Sonde ist allerdings leitend und muss nicht neutral geladen sein \cite{LPfunction, LPBochum, IU_LP}. Ihr Potential ist variabel und wird immer relativ zum Plasmapotential angegeben. Wird ein positives Potential angelegt, werden aktiv Elektronen aus dem Plasma gezogen und Ionen abgestoßen, es handelt sich um den sogenannten Elektronensättigungsbereich. Wird das Potential gesenkt, befindet sich die Sonde zunächst im Elektronenanlaufstrombereich. Das heißt, dass noch Elektronen bis zur Sonde vordringen, zeitgleich aber schon Ionen angezogen werden, sodass der gemessene Nettostrom sinkt. Der Elektronenanlaufstrombereich ist gegeben, solange das Potential der Sonde noch höher ist als das Floating Potential. Dieses entspricht dem an der Sonde anliegenden Potential, bei dem kein Nettostrom mehr gemessen wird, weil sich Elektronen- und Ionenflüsse ausgleichen. Sinkt das Potential noch weiter und unterschreitet das Floating Potential, werden vornehmlich Ionenströme gemessen, da die Mehrheit der Elektronen von der Drahtspitze fern gehalten wird. Es wird vom sogenannten Ionensättigungsbereich gesprochen. Der Verlauf dieser \(I(U)\)-Kennlinie ermöglicht Rückschlüsse auf entsprechende Plasmaparameter. Der exponentielle Anstieg der Kurve im Elektronenanlaufstrombereich gibt beispielsweise Auskunft über die Energieverteilung der Elektronen und somit über deren Temperatur sowie die Elektronendichte. Eine idealisierte \(I(U)\)-Kennlinie ist in Abb. \ref{fig:IULP} zu sehen.\\

\begin{figure}[!htb] 
	\centering
	\includegraphics[scale=0.45]{figures/IU_LP.png}
	\caption{\(I(U)\)-Kennlinie einer idealen Langmuir Sonde aus \cite{IU_LP}. Für an der Langmuir Sonde angelegte Potentiale, welche kleiner als das Floating Potential \(V_f\) sind, befindet sich die Sonde im Ionensättigungsbereich. Für ein Potential zwischen \(V_f\) und dem Elektronensättigungspotential \(V_s\) wird vom Elektronenanlaufstrombereich gesprochen. Die \(I(U)\)-Kennlinie steigt hier exponentiell an. Der Elektronensättigungsbereich beginnt bei einem Sondenpotential von \(V_s\). Hier steigt der gemessene Nettostrom nur noch sehr langsam.}\label{fig:IULP}
\end{figure}

Zu Langmuir Sonden ist anzumerken, dass sie nicht nur gewollte Wechselwirkungen mit dem Plasma ausführen \cite{LPfunction}(S. 114ff). Ihre Anwesenheit im Plasma verändert dieses lokal, sodass Messergebnisse verfälscht werden. Sie können des Weiteren die Elektronentemperatur systematisch unterschätzen, weil Sekundärelektronen ebenso gemessen werden wie Plasmaelektronen. Sekundärelektronen sind Elektronen, die durch den Beschuss des Drahts mit Photonen, Elektronen und Ionen frei werden. Außerdem ist zu beachten, dass die Sonden innerhalb von elektrischen Sheaths ungeeignete Messgeräte sind.\\

Diese Fehlerquellen sollten beachtet werden, wenn mit Langmuir Sonden und ihren Daten gearbeitet wird. Zumindest das Problem mit Messungen im elektrischen Sheath konnte für diese Arbeit jedoch vernachlässigt werden, da sich die Drahtspitzen mit \qty{5}{mm} \cite{ArunLP} über dem Divertor außerhalb des nur wenige Zehntel Millimeter hohen Langmuir Sheaths befanden. Das dem Langmuir Sheath vorgelagerte Presheath ist weiter ausgedehnt, beeinträchtigte die Funktion der Langmuir Sonden aber nur minimal, da in diesem Bereich der Potentialunterschied zum Plasma nur noch sehr klein ist \cite{PWI-Dirk}.\\

An dieser Stelle wird nun im Besonderen nochmal auf die Langmuir Sonden in W7-X eingegangen. Seit OP2 hat Wendelstein 7-X einen wassergekühlten HHF-CFC-Divertor, der thermischen Belastungen bis \qty[per-mode = symbol]{10}{\MW\per\m\tothe{2}} standhält und somit für lange Entladungen bis \qty{30}{min} geeignet ist \cite{W7XtechDetailsOP2}. Die Temperatur, bei welcher kohlenstoffbasiertes Divertormaterial sublimiert, liegt trotzdem noch unter dem Schmelzpunkt der aus Wolfram gefertigten Drähte der Langmuir Sonden \cite{ArunLP}. Dennoch sind die Langmuir Sonden im Gegensatz zum HHF-CFC-Divertor nicht für \qty{30}{min} Dauerbetrieb geeignet. Das ist unter anderem darauf zurückzuführen, dass sie mit einem Durchmesser von \(\approx\) \qty{2}{mm} im Gegensatz zum Divertor zu klein sind, um mit Wasser gekühlt zu werden. Zusätzlich wird durch das Einführen ins Plasma ohnehin eine deutlich höhere Belastung von \qty[per-mode = symbol]{100}{\MW\per\m\tothe{2}} bis \qty[per-mode = symbol]{200}{\MW\per\m\tothe{2}} erwartet \cite{ArunLP}. Eine längere Aufenthaltszeit im Plasma würde daher zweifelsohne zum Schmelzen der Sonden führen, was deren Zerstörung und außerdem die Verunreinigung des Plasmas mit Wolfram bedeutet. Um dieses Szenario zu vermeiden, handelt es sich bei den in W7-X montierten Messgeräten um sogenannte Pop-up Langmuir Sonden, die in Intervallen ins Plasma ein- und ausgefahren werden. Die Zeit im Plasma ist auf höchstens \qty{50}{ms} begrenzt, die Zeit außerhalb des Plasmas dient der Abkühlung der Sonden. Dies führt zu einer diskontinuierlichen Messung mit einer durchschnittlichen Datenerhebung aller \qty{2}{s} bis \qty{3}{s}.\\

Von besonderem Interesse für eine Datenerfassung mittels Langmuir Sonden wurden die zwei Bereiche des Divertors ausgemacht, in denen sich die Strikeline am häufigsten befindet: Der low-iota Bereich gegeben durch Targetelement TM2h07 und TM3h01 sowie der high-iota Abschnitt repräsentiert durch TM8h01 (Abb. \ref{fig:DivertorLP}). Dabei war auch die Möglichkeit, überhaupt Langmuir Sonden montieren zu können, ausschlaggebend, da das Kühlungssystem des Divertors den Raum für Neuinstallationen beschränkt \cite{ArunLP}. Auf TM2h07 sind sechs Langmuir Sonden angebracht, die radial in einer Linie vom Pumpspalt weg angeordnet sind. Die acht Sonden auf TM3h01 sind weiter vom Pumpspalt entfernt, aber ebenso radial ausgerichtet wie die in TM2h07. Auf TM8h01 sind vier Sonden in zunehmenden Abstand zum Pumpspalt befestigt. Die Messpunkte sind in der oberen und unteren Divertoreinheit von Modul 5 an zueinander symmetrischen Positionen, sodass insgesamt 36 Messstellen zur Verfügung stehen. Die toroidalen Winkel sind in\linebreak Tab. \ref{tab:phiLP} angegeben, die Abstände vom Pumpspalt in Tab. \ref{tab:MFieldAngle}.\linebreak Abb. \ref{fig:LPPositions} veranschaulicht die Positionen der Langmuir Sonden auf den Targets. Es sind nicht immer alle Sonden aktiv - zum Teil werden nur jene betrieben, in deren Bereich die Strikeline für die entsprechende Entladung fällt.

\section{Helium-Beam-Diagnostik}
Um die Zuverlässigkeit der Messdaten der Langmuir Sonden abschätzen zu können, werden ihre Messergebnisse für ausgewählter Entladungen mit denen der Helium-Beam-Diagnostik verglichen. Es handelt sich um die Entladungen 20250508.071, 20250507.009, 20250507.007 und 20250320.077, die Benennung folgt dem Schema YYYYMMDD.ID, wobei ID die Nummer der Plasmaentladung am jeweiligen Tag ist.\\

Die Helium-Beam-Diagnostik beruht auf dem Prinzip der Spektroskopie und misst im Bereich des Plasmarandes über der oberen Divertoreinheit von Modul 5 bei einem toroidalen Winkel von \qty{300,3}{\degree} \cite{HeBeam, PPFoisal}. Das ist nahe der Position der Langmuir Sonden, die in dieser Divertoreinheit bei einem toroidalen Winkel von \qty{298,5}{\degree} montiert sind \cite{ArunLP}. Die Helium-Beam-Diagnostik läuft nicht kontinuierlich mit, da Helium-Gasstöße aus bestimmten Gasventilen benötigt werden, damit sich eine entsprechende Anzahl von He-Teilchen im Bereich der Lines of Sight (LOS) des Spektrometers befindet \cite{HeBeam, PPFoisal}. Die Verhältnisse der Intensitäten \(I_{\lambda_1}/I_{\lambda_2}\) und \(I_{\lambda_3}/I_{\lambda_2}\) der verschiedenen Spektrallinien \(\lambda_1\), \(\lambda_2\) und \(\lambda_3\) hängen dabei von \(T_e\) und \(n_e\) ab. Die Spektrallinie der Wellenlänge \(\lambda_1\) (\qty{667,8}{nm}) entsteht beim Übergang eines Elektrons vom Energieniveau 3\(^1\)D zu 2\(^1\)P. Photonen der Wellenlänge \(\lambda_2\) (\qty{728,1}{nm}) werden beim Elektronenübergang vom Energieniveau 3\(^1\)S zu 2\(^1\)P frei, während Licht der Wellenlänge \(\lambda_3\)
(\qty{706,5}{nm}) beim Übergang eines Elektrons vom Energieniveau 3\(^3\)S auf 2\(^3\)P ausgesendet wird.\\

Generell stimmen die Messungen der Elektronentemperatur beider Diagnostiken gut überein. Die Elektronendichte hingegen wird von den Langmuir Sonden im Vergleich zur mit weiteren Diagnostiken übereinstimmenden Helium-Beam-Diagnostik tendenziell unterschätzt \cite{PPFoisal}.   

\begin{figure}[!htb] 
	\centering
	\subfigure[Low-iota Bereich]{\includegraphics[width=0.6\textwidth]{figures/LPPositionTM2h.png}}
	\subfigure[High-iota Bereich]{\includegraphics[width=0.6\textwidth]{figures/LPPositionTM8h.png}}
	\caption{Positionen der Langmuir Sonden auf den Targetelementen im low-iota (TM2h07, TM3h01) und high-iota (TM8h01) Bereich einer Divertoreinheit markiert durch rote Kreuze. Der Koordinatenursprung liegt am Pumpspalt.}\label{fig:LPPositions}
\end{figure}

\section{Infrarot-Kamerasysteme}
Die Oberflächentemperatur des Divertors und anderer PFCs von W7-X wird durchgängig, das heißt in allen Modulen und zu allen Zeiten des Experimentalbetriebs, durch Infrarotkameras beobachtet \cite{Protection}. Das dient dem Schutz vor lokaler Überhitzung und den daraus resultierenden Schäden an Bauteilen und dem Stellarators im Allgemeinen. Die Oberflächentemperatur der PFCs wird dabei nach dem folgenden Messprinzip bestimmt.\\

Thermographische Kameras, welche sensitiv für Licht im infraroten Spektrum sind, ermöglichen die Messung der Oberflächentemperatur des Divertors unter Nutzung des Planck'schen Strahlungsgesetzes \cite{Protection}. Dieses besagt, dass jeder Körper, in Abhängigkeit seiner Oberflächentemperatur und Emissivität, Lichtspektren mit unterschiedlicher Intensitätsdistribution über die verschiedenen Wellenlängen aussendet. Formel \ref{eq:planck} beschreibt die Energiedichte \(U\) in Bezug auf Frequenz \(f\) und die Oberflächentemperatur \(T_s\), wobei \(k_B\) die Boltzmann Konstante, \(c\) die Lichtgeschwindigkeit im Vakuum und \(h\) das Planck'sche Wirkungsquantum ist \cite{planck}:

\begin{align}
	U(f, T_s) = \frac{8 \pi h f^3}{c^3 \cdot \left(\exp\left(\frac{h f}{k_B T_s}\right) - 1\right)}\label{eq:planck}\text{\quad .}
\end{align}

Für die in W7-X erreichten Oberflächentemperaturen hat die Energiedichteverteilung ihr Maximum im infraroten Spektrum \cite{Protection}. Wird zusätzlich die Emissivität des Objekts mit einbezogen, so kann aus dem gemessenen Photonenfluss die Oberflächentemperatur errechnet werden. Konkret wird das in Wendelstein 7-X durch Bolometer umgesetzt. Einfallende Photonen treffen die Oberfläche des Messgeräts und heizen diese auf. Das wird durch eine Matrix von Widerstandsthermometern registriert, von denen jedes einzelne einem Kamerapixel entspricht \cite{IRCamCaleo, bolometer}.\\

Zur Berechnung der erosionsbezogenen Größen ist die Kenntnis der Oberflächentemperatur an den Positionen nötig, an denen die Langmuir Sonden angebracht sind. Das ist in Modul 5 in der oberen und unteren Divertoreinheit auf TM2h07, TM3h01 und TM8h01. Die Daten der oberen Einheit liefert die Infrarotkamera in Port AEF51, für die unteren Divertoreinheit ist es die Kamera in Port AEF50. Obwohl die Temperaturdaten für den gesamten Divertor abrufbar sind, wird nicht TM2h07 und TM3h01 untersucht, sondern auf die benachbarten TM2h06 und TM3h02 ausgewichen. Genauso wird statt TM8h01 TM8h02 betrachtet. Der Grund dafür ist, dass zwischen TM2h07 und TM3h01 beziehungsweise TM7h06 und TM8h01 ein Spalt ist, weil dort das Targetmodul von TM2h zu TM3h beziehungsweise TM7h zu TM8h wechselt. Dieser Spalt mit den abgeschrägten Targetmodulkanten führt zum Entstehen von sogenannten Leading Edges und Schattenzonen \cite{MfieldIncidentAngle}.\\

Leading Edges am Divertor von W7-X sind verstärkt erwärmte Regionen, in denen das von einer Seite kommende Plasma die abgeschrägte Kante des Targetmoduls mit einem steileren Winkel (näher an der Oberflächennormalen) trifft \cite{MfieldIncidentAngle}. Demnach wird die Energie der auftreffenden Teilchen auf eine kleinere Fläche verteilt, die dann höhere Temperaturen erreicht. Zugleich trifft eben jenes Plasma das zweite, im Schatten liegende Targetmodul an dessen abgeschrägter Kante in flacherem Winkel, sodass die Energie der auftreffenden Teilchen über eine größere Fläche verteilt wird. Die Temperatur dieses Bereichs ist somit geringer. Beides verfälscht das Ergebnis der Temperaturmessung. Das Targetelement mit Leading Edge wird als durchschnittlich zu heiß angenommen, obgleich nur der Randbereich stärker erhitzt wird, nicht das gesamte Target. Umgekehrt ist das Targetelement in der Schattenzone im Durchschnitt zu kalt, weil der Randbereich kühler ist. Damit diese Ungenauigkeiten nicht das Ergebnis der Rechnungen rund um die Erosionsprozesse beeinträchtigen, werden die benachbarten Targetelemente betrachtet. Diese sind nicht von Leading Edges oder Schattenzonen betroffen.\\

Auch gibt es auf den gewählten Flächen keine prominenten Surface Layers. Das sind Ablagerungen von Kohlenstoffatomen auf dem Divertor, wobei sich die entstehende chemische Struktur deutlich vom Rest des Divertors unterscheidet \cite{IRCamCaleo}. Surface Layers haben eine schlechte thermische Verbindung zum Divertor und werden deshalb schnell sehr heiß, weil kaum Wärme abgeführt werden kann. Sie sind deshalb auf Infrarot-Kamerastreams besonders auffällig und spiegeln keinesfalls die reale Divertortemperatur wieder.

\section{Spektroskopie und Restgasanalyse zur Bestimmung der Ionenkonzentrationen}\label{sec:IonDiag}
Die Bestimmung der Konzentration einer bestimmten Art von Verunreinigung wird am Plasmarand von Wendelstein 7-X nicht routinemäßig durchgeführt. Es gibt jedoch eine Anzahl an Diagnostiken, die Rückschlüsse auf diese physikalische Größe zulassen. Die zur Bestimmung der Konzentrationen notwendigen Berechnungen sind jedoch zumeist mit erheblichem Rechenaufwand verbunden, wenn mehr als eine qualitative Aussage (im Plasma enthalten oder nicht) benötigt wird. Deshalb kann nur exemplarisch untersucht werden, wie hoch die Konzentration von Sauerstoff- und Kohlenstoffionen in einzelnen Entladungen ist. Die für diese Analyse in Frage kommenden Diagnostiken werden in diesem Abschnitt vorgestellt, obgleich die im Zuge dieser Arbeit erhaltenen Daten schlussendlich keine Bestimmung der Konzentrationen von C- und O-Ionen zuließen.\\

Die erste Diagnostik ist Charge Exchange Recombination Spectroscopy (CXRS), sie liefert ein radiales Profil der benötigten \(n_i\) Werte \cite{CXRS, IPPcxrs}. Damit Daten bereitgestellt werden können, muss Neutral Beam Injection (NBI) als Heizquelle (mit) eingesetzt werden. Demnach sind immer mindestens Helium und Argon im Plasma enthalten. Wenn die injizierten Neutralteilchen ins Plasma gelangen, kommt es dort zu Wechselwirkungen mit den Plasma- und Verunreinigungsionen, wobei ein Elektron vom Neutralteilchen zum Ion übertragen wird. Das Ion, welches das Elektron aufgenommen hat, ist zunächst im angeregten Zustand. Das Elektron springt dann jedoch unter Aussendung eines Photons charakteristischer Wellenlänge auf ein niedrigeres Energieniveau zurück. Diese Strahlung kann durch CXRS analysiert werden und erlaubt unter anderem Rückschlüsse auf die Konzentrationen der im Plasma enthaltenen Verunreinigungsionen.\\

Die zweite Möglichkeit zur Bestimmung von \(f_O\) und \(f_C\) ist die Restgasanalyse (Diagnostic Residual Gas Analyzer, DRGA) \cite{DRGA}. Sie untersucht abgepumptes Gas mithilfe eines Massenspektrometers auf seine Bestandteile. Der DRGA ist mehrere Meter vom Plasma entfernt, damit das Magnetfeld, welches das Plasma einschließt, die Massenspektroskopie nicht tangiert. Das ist insofern problematisch, weil nur Teilchen analysiert werden, die es bis zur DRGA-Diagnostik schaffen. Diese müssen zunächst am Divertor neutralisiert werden, um dann durch den Pumpspalt und im Anschluss an der Cryo-Vakuum-Pumpe vorbei in den Pumpstutzen der Diagnostik zu gelangen. Danach folgt eine über \qty{7}{m} lange Leitung bis zur Diagnostik selbst. Zusätzlich zum Ziehen eines Rückschlusses auf die Konzentration am Plasmarand muss bei der Auswertung der Analyse beachtet werden, dass die entsprechende Verunreinigung auch im Molekül gebunden sein kann, beispielsweise Kohlenstoff in CH\(_4\). Außerdem gibt es eine zeitliche Verzögerung der gelieferten Messwerte der Diagnostik von etwa \qty{1}{s}, die auf die Dauer des Transports des Teilchens aus dem Plasma bis zum DRGA entfällt. Wenn also verwertbare Daten generiert werden sollen, muss die Entladung über ein längeres Zeitintervall stabile Plasmaparameter aufweisen, damit die träge DRGA reagieren kann. Die Messung funktioniert weiterhin am zuverlässigsten bei Entladungen in der Standardkonfiguration des Magnetfeldes sowie bei ausgeschalteter (warmer) Cryo-Vakuum-Pumpe. Außerdem ist DRGA bei der Analyse von Edelgasen genauer als bei der Betrachtung von Kohlenstoff und Sauerstoff.\\

Als letzte, in dieser Arbeit untersuchte Option kann auf die Messungen von HEXOS (High Efficiency XUV Overview Spectrometer) zurück gegriffen werden \cite{HEXOSdesign, HEXOScalibration}. Das Spektroskop untersucht das Vorkommen verschiedener Ionenspezies im Core Plasma anhand ihrer Vakuum-UV-Strahlung, wobei aufgrund fehlender Kalibration jedoch nur qualitative Aussagen getroffen werden können. Allerdings ist der Vergleich zweier ähnlicher Entladungen möglich, sodass für eine Verunreinigungsart ein zeitlicher Trend erstellt werden kann. HEXOS läuft bei jeder Entladung mit, ist allerdings je nach Ionenspezies unterschiedlich sensitiv. C-II und O-III Spektrallinien fallen in einen Bereich großer Ungenauigkeit.\\

Bei der Auswahl der Entladungen zur Bestimmung der Verunreinigungskonzentrationen sind neben den Anforderungen zur Betreibung der Diagnostiken (laufende NBI, abgeschaltene Cryo-Vakuum-Pumpe, stabile Plasmaparameter in einer Entladung in EIM) noch weitere Kriterien ausschlaggebend. Es ist beispielsweise auf besondere Vorkommnisse - wie das Schmelzen einer Sonde - zu achten, die die Messergebnisse durch zusätz\-lichen Eintrag von Verunreinigungen verfälschen könnten. Aus diesem Grund sollten keine Entladungen kurz nach derartigen Ereignissen zur Analyse der Verunreinigungskonzentration gewählt werden. Zusätzlich muss der zeitliche Abstand zu den Borierungen in Betracht gezogen werden, da diese die Konzentration von Verunreinigungsionen im Plasma reduzieren \cite{MarkusOP1.2a, MarkusOP1.2b}. Es ist aus diesem Grund wichtig zu wissen, wie sich die Konzentrationen von Verunreinigungen am Plasmarand zwischen zwei Borierungen ändern. In OP2.2 wurde monatlich, in OP2.3 aller zwei Monate boriert, die konkreten Tage sind in Tab. \ref{tab:Bor} vermerkt. Zuletzt ist es von Vorteil, wenn für die gewählten Entladungen auch die Daten von möglichst vielen Langmuir Sonden und Infrarotkameras zu möglichst vielen Zeitpunkten vorliegen. Das ermöglicht die direkte Verknüpfung aller Daten für diese Entladung.\\

Basierend auf den genannten Kriterien wurden die Entladungen 20250304.075,\linebreak 20250408.055 und 20250408.079 zur Analyse durch CXRS und DRGA ausgewählt. Die Benennung der Entladungen folgt dem Schema YYYYMMDD.ID, wobei ID die Nummer der Plasmaentladung am jeweiligen Tag ist. Für die Untersuchung mit HEXOS wurden 17 zu verschiedenen Zeitpunkten der beiden Kampagnen durchgeführte Referenzentladungen betrachtet. Keine der genannten Optionen führte jedoch zur erfolgreichen Bestimmung der Ionenkonzentrationen von C- und O-Ionen. Stattdessen wurden drei Sets von Konzentrationen basierend auf Messdaten der effektiven Ladungszahl des Plasmas angenommen. Der Versuch der Konzentrationsbestimmung mittels CXRS, DRGA und HEXOS sowie die Begründung der angenommenen Ionenkonzentrationen wird in Abschnitt \ref{sec:IonParam} genauer dargelegt.\\

