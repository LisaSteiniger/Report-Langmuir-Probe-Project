\chapter{Langmuir Sonden}\label{chap:theory}
Um das Verhalten des Plasmas während einer einzelnen Entladung beobachten zu können, ist eine Vielzahl unterschiedlicher Diagnostiken in W7-X verbaut, die die Bestimmung verschiedener Plasmaparameter erlaubt. Zu diesen Diagnostiken zählen unter anderem die im Plasmagefäß angebrachten Langmuir Sonden (LS).\\

Auf den Wandelementen von W7-X – konkret auf dem Divertor in Modul 5 (M5) – sind 36 LS montiert. Sie befinden sich an symmetrischen Positionen der oberen und unteren Divertoreinheit (DU, je 18 LS pro DU) im high- und low-iota Bereich (4 LS bzw. 14 LS) und ermöglichen dort unter anderem die Messung von Elektronendichte und Elektronentemperatur. Aufgrund der hohen thermischen Lasten von \qty[per-mode=symbol]{100}{\MW\per\m\tothe{2}} – \qty[per-mode=symbol]{200}{\MW\per\m\tothe{2}} können die Sonden dem Plasma nicht dauerhaft ausgesetzt sein, da das Material schmelzen und die Sonden zerstört würden. Stattdessen handelt es sich um sogenannte Pop-up Langmuir Sonden, die in Intervallen ins Plasma ein- und ausgefahren werden. Das resultiert in einer diskontinuierlichen Messung etwa aller 2-3 s.\\

Die Funktionsweise einer LS ist die Folgende: Die Stromstärke an der Probe wird in Abhängigkeit der an der Sonde angelegten Spannung gemessen, welche zwischen -180 V und 20 V variiert wird. Die Bestimmung der Elektronendichte und -temperatur basiert dann auf der Auswertung der Stromstärke-Spannungskurve. Im Normalfall ist diese im Elektronenanlaufstrombereich – dem für die
Bestimmung relevanten Teil der Kurve – exponentiell.\\