\chapter{Zusammenfassung}\label{chap:summary}
Wendelstein 7-X ist ein Experiment vom Typ Stellarator, dessen Plasmaeinschluss auf einem toroidalen Magnetfeld mit poloidaler Verdrillung beruht. Dieses kann in verschiedenen Magnetfeldkonfigurationen auftreten und weist abhängig davon unterschiedlich viele magnetische Inseln mit variierenden Positionen auf. Sie sind vom Core Plasma separiert, ihre Magnetfeldlinien werden vom High Heat Flux Carbon Fibre Composite Divertor geschnitten. Die Position dieser Schnittstelle, der sogenannten Strikeline, hängt - wie die der Inseln selbst - von der Magnetfeldkonfiguration ab. Die Hauptgruppen sind die Low-Iota Konfiguration (DBM), die Standardkonfiguration (EIM) und die High-Mirror Konfiguration (KJM) mit einer Strikeline im low-iota Bereich des Divertors sowie die High-Iota Konfiguration (FTM) mit einer Strikeline im high-iota Bereich des Divertors. Die Strikeline ist die Region mit der ausgeprägtesten Wechselwirkung zwischen Plasma und Divertor, da dort die den geschnittenen Magnetfeldlinien folgenden geladenen Teilchen auftreffen.\\

Durch ihre Energie und Reaktivität haben die auftreffenden Ionen einen erodierenden Effekt auf den Divertor. Das abgetragene Material wird zum Teil direkt wieder abgelagert, zum Teil gelangt es als Verunreinigung ins Plasma. Dort kann das erodierte Material zur erhöhten Abstrahlung von Energie führen. Außerdem begrenzt die Erosion die Lebensdauer des Divertors.\\

Das Verständnis der in W7-X ablaufenden Erosionsprozesse (erosion-deposition) ist daher essentiell zur Optimierung des Divertors und der Kontrolle der Verunreinigungskonzentrationen im Plasma. Diese Arbeit untersuchte aus diesem Grund die Ero\-sion durch Wasserstoff, Kohlenstoff und Sauerstoff im low-iota und high-iota Bereich des HHF-CFC-Divertors von W7-X in OP2.2 und OP2.3. Dazu wurden Bruttoersion, Bruttodeponierung und Nettoerosion aus den Messwerten der Elektronentemperatur, der Elektronendichte und Oberflächentemperatur des Divertors berechnet, welche von Langmuir Sonden und Infrarot-Kamerasystemen bereitgestellt wurden. Zusätzlich zu den Messwerten wurden Einfallswinkel der Magnetfeldlinien und Ionen auf den Divertor sowie Ionenkonzentrationen von Wasserstoff H\(^+\), Kohlenstoff C\(^{2+}\) und Sauerstoff O\(^{3+}\) abgeschätzt. Durch Hochrechnungen ergab sich eine nettoerodierte Schichtdicke für die Kampagnen OP2.2 und OP2.3.\\

Die durchschnittliche Elektronendichte bewegte sich zwischen \qty[per-mode=symbol]{1,12\pm0,25E+18}{\per\m\tothe{3}} und \qty[per-mode=symbol]{2,79\pm0,01E+19}{\per\m\tothe{3}} in OP2.2 und zwischen \qty[per-mode=symbol]{4,58\pm0,04E+17}{\per\m\tothe{3}} und \qty[per-mode=symbol]{8,64\pm1,53E+18}{\per\m\tothe{3}} in OP2.3. Für die mittlere Elektronentemperatur reichten die Werte in OP2.2 von \qty{5,99\pm2,60}{eV} bis \qty{24,47\pm0,30}{eV} und in OP2.3 von \qty{8,49\pm0,21}{eV} bis \qty{28,42\pm2,70}{eV}. Die gemittelte Oberflächentemperatur des Divertors bewegten sich in OP2.2 im Rahmen von \qty{310}{K} bis \qty{436}{K} und von \qty{312}{K} bis \qty{344}{K} in OP2.3.\\

Die durchschnittlichen Nettoerosionsraten lagen in der gesamten OP2.2 bei maximal \qty[per-mode=symbol]{1,29\pm0,03}{\nm\per\s}. Bei einer Plasmazeit von \qty{20371}{s} resultierte das in einer nettoerodierten Maximalschichtdicke von \qty{0,026\pm0,0005}{mm}. In OP2.3 ergaben sich\linebreak \qty[per-mode=symbol]{0,83\pm0,01}{\nm\per\s} als Maximum der durchschnittlichen Nettoersoionsrate der Kampagne, was bei \qty{28329}{s} Plasmazeit eine maximal abgetragene Schichtdicke von\linebreak \qty{0,024\pm0,0003}{mm} bedeutete. In beiden Kampagnen wurde dieser Maximalabtrag im Bereich der Strikeline der Standardkonfiguration beobachtet, was dem low-iota Bereich des Divertors und einem Abstand von \qty{0,13}{m} vom Pumpspalt entspricht. Dieser Verlauf spiegelte sich auch in den durchschnittlichen Plasmaparametern der Kampagnen wieder. Wird aus den abgetragenen Schichtdicken auf die Masse des nettoerodierten Kohlenstoffstaubs pro Quadratmeter mit dem Plasma in Kontakt stehender Fläche geschlossen, so wurde für OP2.2 ein Wert von \qty{30,66\pm12,61}{g} erreicht. Für OP2.3 waren es \qty{60,15\pm23,93}{g}.\\

Plasmaparameter und maximale Nettoerosionsraten waren in OP2.2 höher, durch die längere Plasmazeit in OP2.3 waren die maximal nettoerodierten Schichtdicken jedoch in beiden Kampagnen ähnlich. Trotzdem war die nettoerodierte Masse von Kohlenstoff in OP2.3 durch die höheren nettoerodierten Schichtdicken auch abseits der Strikeline der Standardkonfiguration höher als in OP2.2. Zwar wurden die Parameter- und Schichtdickenverläufe beider Kampagnen von der Standardkonfiguration dominiert, spiegelten in OP2.3 aber auch die im Mittelbereich der low-iota Region des Divertors erhöhten Plasmaparameter und Schichtdicken der Low-Iota Konfiguration wieder. Das passt zur Zusammensetzung der Plasmazeit, die in OP2.2 zu \qty{56,35}{\%} und in OP2.3 zu \qty{51,06}{\%} aus EIM bestand. DBM war mit \qty{18,26}{\%} in OP2.3, aber nur \qty{0,48}{\%} in OP2.2 vertreten. KJM und FTM hatten weniger Einfluss. Das ist darauf zurückzuführen, dass ihre Strikeline nicht im low-iota Bereich auf dem horizontalen Target liegt, wo die meisten Daten für \(n_e\) und \(T_e\) erfasst wurden. Die High-Mirror Konfiguration hat ihre Strikeline zwar im low-iota Bereich des Divertors, jedoch primär auf dem vertikalen Target. Die High-Iota Konfiguration hat ihre Strikeline hingegen in der high-iota Region des Divertors, deren Erosionsprozesse wegen ihrer schlechteren Datenlage weniger aussagekräftig waren.\\ 

Der Vergleich der Nettoerosionsraten mit früheren Abschätzungen aus OP1.2b zeigte Übereinstimmungen in Verlauf und Werten. Die maximale Nettoerosionsrate an der Strikeline der Standardkonfiguration lag in OP1.2b bei \qty[per-mode=symbol]{1,1}{\nm\per\s} bis \qty[per-mode=symbol]{2,5}{\nm\per\s}. Auch die nettoerodierte Masse an Staub lieferte ähnliche Ergebnisse wie andere Hochrechnungen, die \qty{48,9}{g} in OP2.2 und \qty{66,7}{g} in OP2.3 prognostiziert hatten. Ein Vergleich mit experimentell bestimmten Werten war nicht möglich, da diese nicht vorliegen.\\

Generell passen die Ergebnisse dieser Arbeit gut zu den Erwartungen und weichen nur wenig von anderen Abschätzungen ab. Dennoch ist durch die vielen nötigen Annahmen und Intra- sowie Extrapolationen die Unsicherheit in den Ergebnissen groß. Genauere Untersuchungen zur exakteren Festlegung von Inputparametern wie der Ionenkonzentration würden die Belastbarkeit der Ergebnisse erhöhen.  