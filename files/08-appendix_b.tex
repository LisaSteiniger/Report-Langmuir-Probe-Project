\chapter{Appendix C: Tabellenwerte}\label{chap:appendixB}
\section{Parameter zur Berechnung der Zerstäubungsausbeute}
\begin{table}[!htb]
	\caption{Werte für den Parameter \(Q_y\) zur Berechnung der physikalischen Zerstäubungsausbeute \cite{PWI-Dirk}(S. 81). Die Zeile gibt das Material an, aus welchem das Target besteht, die Spalte das auftreffende Ion. Selbstzerst. kürz Selbstzerstäubung ab.}\label{tab:Qy}
	\centering
	\begin{tabular}{|c||r|r|r|r|r|r|}\hline
		Target/Ion&Wasserstoff&Deuterium&Tritium&Helium&Selbstzerst.&Sauerstoff\\\hline\hline
		Beryllium&0,07&0,11&0,14&0,28&0,67&-\\\hline
		Kohlenstoff&0,05&0,08&0,10&0,20&0,75&1,02\\\hline
		Eisen&0,07&0,12&0,16&0,33&10,44&-\\\hline
		Molybdän&0,05&0,09&0,12&0,24&16,27&-\\\hline
		Wolfram&0,04&0,07&0,10&0,20&33.47&-\\\hline
	\end{tabular}
\end{table}

\begin{table}[!htb]
	\caption{Werte für den Parameter \(E_{TF}\) in (\unit{eV}) zur Berechnung der physikalischen Zerstäubungsausbeute \cite{PWI-Dirk}(S. 81). Die Zeile gibt das Material an, aus welchem das Target besteht, die Spalte das auftreffende Ion. Selbstzerst. kürz Selbstzerstäubung ab.}\label{tab:ETF}
	\centering
	\begin{tabular}{|c||r|r|r|r|r|r|}\hline
		Target/Ion&Wasserstoff&Deuterium&Tritium&Helium&Selbstzerst.&Sauerstoff\\\hline\hline
		Beryllium&256&282&308&720&2.208&-\\\hline
		Kohlenstoff&415&447&479&1.087&5.688&9.298\\\hline
		Eisen&2.544&2.590&2.635&5.517&174.122&-\\\hline
		Molybdän&4.719&4.768&4.817&9.945&533.127&-\\\hline
		Wolfram&9.871&9.925&9.978&20.376&1.998.893&-\\\hline
	\end{tabular}
\end{table}

\begin{table}[!htb]
	\caption{Werte für die Schwellenenergie der physikalischen Zerstäubung \(E_{th}\) in (\unit{eV}) zur Berechnung der physikalischen Zerstäubungsausbeute \cite{PWI-Dirk}(S. 81). Die Zeile gibt das Material an, aus welchem das Target besteht, die Spalte das auftreffende Ion. Selbstzerst. kürz Selbstzerstäubung ab.}\label{tab:EthPhys}
	\centering
	\begin{tabular}{|c||r|r|r|r|r|r|}\hline
		Target/Ion&Wasserstoff&Deuterium&Tritium&Helium&Selbstzerst.&Sauerstoff\\\hline\hline
		Beryllium&13&13&15&16&24&-\\\hline
		Kohlenstoff&31&28&30&32&53&61,54\\\hline
		Eisen&61&32&23&20&31&-\\\hline
		Molybdän&172&83&56&44&49&-\\\hline
		Wolfram&447&209&136&102&62&-\\\hline
	\end{tabular}
\end{table}

\begin{table}[!htb]
	\caption{Werte für die Sublimationswärme \(E_s\) in (\unit{eV}) zur Berechnung der physikalischen Zerstäubungsausbeute \cite{PWI-Dirk}(S. 81).}\label{tab:Es}
	\centering
	\begin{tabular}{|r|r|r|r|r|}\hline
		Beryllium&Kohlenstoff&Eisen&Molybdän&Wolfram\\\hline\hline
		3,38&7,42&4,34&6,83&8,68\\\hline
	\end{tabular}
\end{table}

\begin{table}[!htb]
	\caption{Werte für die Fitparameter für die Fitfunktionen \(f\), \(b\) und \(c\) zur Berechnung der physikalischen Zerstäubungsausbeute mit\\\(Fitfunktion = y0 + \sum_i A_i \cdot \exp\left(-\frac{E - x0}{t_i}\right)\) \cite{BehrischEckstein}.}\label{tab:fitMarkus1}
	\centering
	\begin{tabular}{|c||r|r|r|}\hline
		Parameter/Funktion&\multicolumn{1}{c|}{f}&\multicolumn{1}{c|}{b}&\multicolumn{1}{c|}{c}\\\hline\hline
		y0&4,55878&1,22200&0,85257\\\hline
		x0&25,5644&27,59683&37,36542\\\hline
		A1&20,17943&10,24535&-0,10577\\\hline
		t1&29,8123&31,09355&346,95644\\\hline
		A2&12,08692&7,29825&-0,11142\\\hline
		t2&150,66038&185,60025&346,94662\\\hline
		A3&8,99236&4,90847&-0,12915\\\hline
		t3&946,68968&1040,42162&346,92395\\\hline    
	\end{tabular}
\end{table}

\begin{table}[!htb]
	\caption{Werte für den Fitparameter \(Y_0\) zur Berechnung der physikalischen Zerstäubungsausbeute in Abhängigkeit der Teilchenenergie \(E\) in (\unit{eV}) mit \(E_{min} \ll E \le E_{max}\) und \(E'=E/\)\unit{eV} \cite{BehrischEckstein}.}\label{tab:fitMarkus2}
	\centering
	\begin{tabular}{|r|r||r|}\hline
		\(E_{min}\)&\(E_{max}\)&\multicolumn{1}{c|}{\(Y_0\)}\\\hline\hline
		\(-\infty\)&0&0\\\hline
		0&40& -3,318\(\times 10^{-4}\) + 1,167\(\times 10^{-5} \cdot E'\)\\\hline
		40&50&-0,00141 + 3,86\(\times 10^{-5} \cdot E'\)\\\hline
		50&70&-0,0046 + 1,0245\(\times 10^{-4} \cdot E'\)\\\hline
		70&100&-0,01206 + 2,09\(\times 10^{-4} \cdot E'\)\\\hline
		100&140&-0,02231 + 3,115\(\times 10^{-4} \cdot E'\)\\\hline
		140&200&-0,0256 + 3,35\(\times 10^{-4} \cdot E'\)\\\hline
		200&300&-0,019 + 3,02\(\times 10^{-4} \cdot E'\)\\\hline
		300&500&0,005 + 2,22\(\times 10^{-4} \cdot E'\)\\\hline
		500&1000&0,054 + 1,24\(\times 10^{-4} \cdot E'\)\\\hline
		1000&3000&0,1425 + 3,55\(\times 10^{-5}\cdot E'\)\\\hline
		3000&\(\infty\)&0\\\hline
	\end{tabular}
\end{table}

\begin{table}[!t]
	\caption{Tabellenwerte der Fitparameter \(Q_y\) und \(C_d\) sowie der Schwellenenergien \(E_{th}\), \(E_{thd}\) und \(E_{ths}\) in (\unit{eV}) zur Berechnung der chemischen Zerstäubungsausbeute durch Wasserstoff, Deuterium und Tritium \cite{PWI-Dirk}.}\label{tab:ParameterChem}
	\centering
	\begin{tabular}{|c||r|r|r|}\hline
		Parameter/Ion&Wasserstoff&Deuterium&Tritium\\\hline\hline
		\(Q_y\)&0.035&0.1&0.12\\\hline
		\(C_d\)&250&125&83\\\hline
		\(E_{th}\)&31&27&29\\\hline                                                                        
		\(E_{thd}\)&15&15&15\\\hline   
		\(E_{ths}\)&2&1&1\\\hline  
	\end{tabular}
\end{table}

\begin{table}[!t]
	\caption{Werte für die molaren Ionenmassen \(M_i\) in (\unit{u}), die tatsächliche Masse \(m_i\) in (\unit{kg}) ergibt sich aus \(M_i \cdot u = M_i \cdot 10^{-3}/N_A\).}\label{tab:mi}
	\centering
	\begin{tabular}{|r|r|r|r|r|}\hline
		Wasserstoff&Deuterium&Tritium&Kohlenstoff&Sauerstoff\\\hline\hline
		1,008&2,012&3,016&12,011&15,999\\\hline
	\end{tabular}
\end{table}
\clearpage
\section{Daten zu Wendelstein 7-X}
\begin{table}[!htb]
	\caption{Borierung des Plasmagefäßes von Wendelstein 7-X in OP2.2 und OP2.3.}\label{tab:Bor}
	\centering
	\begin{tabular}{|c|c|}\hline
		Operational Phase&Datum der Borierung\\\hline\hline
		\multirow{3}{*}{OP2.2}&13.09.2024\\\cline{2-2}
		&18.10.2024\\\cline{2-2}
		&22.11.2024\\\hline
		\multirow{2}{*}{OP2.3}&14.02.2025\\\cline{2-2}
		&11.04.2025\\\hline
	\end{tabular}
\end{table}

\begin{table}[!htb]
	\caption{Anteile der wichtigsten Magnetfeldkonfigurationen an der Plasmazeit in OP2.2 und OP2.3 von Wendelstein 7-X.}\label{tab:configOP223}
	\centering
	\begin{tabular}{|c|c||r|r|r||r|r|r|}\hline
		\multirow{3}{*}{Konfiguration}&\multirow{3}{*}{Iota}&\multicolumn{3}{c||}{OP2.2}&\multicolumn{3}{c|}{OP2.3}\\\cline{3-8}
		&&Entlad-&\multicolumn{2}{c||}{Plasmazeit}&Entlad-&\multicolumn{2}{c|}{Plasmazeit}\\\cline{4-5}\cline{7-8}
		&&ungen&(s)&(\%)&ungen&(s)&(\%)\\\hline\hline
		DBM&low&17&99&0,48&121&5173&18,26\\\hline
		EIM&standard&989&11478&56,35&1024&14466&51,06\\\hline
		KJM&standard&320&3040&14,92&383&4171&14,72\\\hline
		FTM&high&148&1533&7,52&289&3047&10,76\\\hline\hline
		FMM&high&118&1358&6,67&48&485&1,71\\\hline\hline
		sonstige&-&210&2862&14,05&478&4688&16,55\\\hline
		alle&-&1802&20371&100&2343&28329&100\\\hline
	\end{tabular}
\end{table}

\begin{table}[!htb]
	\caption{Toroidaler Winkel der Positionen der Langmuir Sonden \cite{ArunLP}.}\label{tab:phiLP}
	\centering
	\begin{tabular}{|c|r|}\hline
		Position&\multicolumn{1}{c|}{\(\Phi\) in (\unit{\degree})}\\\hline\hline
		Untere Divertoreinheit: TM2h07, TM3h01&277,6\\\hline
		Obere Divertoreinheit: TM2h07, TM3h01&298,5\\\hline
		Untere Divertoreinheit: TM8h01&303,2\\\hline
		Obere Divertoreinheit: TM8h01&272,9\\\hline
	\end{tabular}
\end{table}

\begin{table}[!t]
	\caption{Einfallswinkel \(\zeta\) der Magnetfeldlinien auf den Divertor von Wendelstein 7-X für low-/standard-/high-iota Konfigurationen an den Positionen der Langmuir Sonden (LS) gemessen von der Targetoberfläche zur Oberflächennormalen. Die Werte von \(\zeta\) basieren auf den durch M. Endler bereitgestellten Tabellen für \(\zeta\) in DBM000+2520, EIM000+2520 und FTM004+2520. Die Position der Sonden ist durch den Abstand vom Pumpspalt definiert und teilt sich nach Divertorabschnitt (DA) in high- und low-iota auf.}\label{tab:MFieldAngle}
	\centering
	\begin{tabular}{|c||c|c|r||r|r|r|}\hline
		LS&Target&DA&Position in (\unit{m})&\(\zeta_{low}\) in (\unit{\degree})&\(\zeta_{std}\) in (\unit{\degree})&\(\zeta_{high}\) in (\unit{\degree})\\\hline \hline
		00&\multirow{6}{*}{TM2h07}&\multirow{14}{*}{low-iota}&0,106&0,420&1,848&3,439\\ \cline{1-1} \cline{4-7}
		01&&&0,132&0,157&1,566&3,136\\ \cline{1-1} \cline{4-7}
		02&&&0,158&0,157&1,230&2,777\\ \cline{1-1} \cline{4-7}
		03&&&0,183&0,416&0,954&2,481\\ \cline{1-1} \cline{4-7}
		04&&&0,209&0,673&0,679&2,188\\ \cline{1-1} \cline{4-7}
		05&&&0,235&0,928&0,408&1,898\\ \cline{1-2} \cline{4-7}\noalign{\vskip\doublerulesep	\vskip-\arrayrulewidth} \cline{1-2} \cline{4-7}
		06&\multirow{8}{*}{TM3h01}&&0,325&1,827&0,547&0,885\\ \cline{1-1} \cline{4-7}
		07&&&0,351&2,072&0,805&0,611\\ \cline{1-1} \cline{4-7}
		08&&&0,377&2,314&1,061&0,341\\ \cline{1-1} \cline{4-7}
		09&&&0,403&2,603&1,365&0,020\\ \cline{1-1} \cline{4-7}
		10&&&0,429&2,842&1,616&0,244\\ \cline{1-1} \cline{4-7}
		11&&&0,454&3,080&1,866&0,506\\ \cline{1-1} \cline{4-7}
		12&&&0,480&3,318&2,115&0,767\\ \cline{1-1} \cline{4-7}
		13&&&0,506&3,556&2,364&1,027\\ \hline \hline
		14&\multirow{4}{*}{TM8h01}&\multirow{4}{*}{high-iota}&0,092&5,198&5,426&5,684\\ \cline{1-1} \cline{4-7}
		15&&&0,117&4,441&4,574&4,725\\ \cline{1-1} \cline{4-7}
		16&&&0,133&4,063&4,150&4,246\\ \cline{1-1} \cline{4-7}
		17&&&0,158&3,435&3,443&3,449\\ \hline
	\end{tabular}
\end{table}