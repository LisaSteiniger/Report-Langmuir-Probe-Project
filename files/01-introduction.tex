\chapter{Einleitung} \label{chap:1}
Fusion - also das Verschmelzen zweier Atomkerne zu einem schwereren Atomkern sowie unter Umständen einem subatomaren Teilchen - stellt eine Möglichkeit der sauberen Energiegewinnung dar, die durch den weltweit steigenden Energiebedarf und die Bedrohung durch den Klimawandel immer mehr Interesse auf sich zieht. Dabei wird die Eigenschaft von Fusionsreaktionen genutzt, dass ein Teil der Masse der Ausgangsstoffe in Energie umgewandelt und abgestrahlt wird. In der Regel wird die Reaktion der Wasserstoffisotope Deuterium und Tritium als am erfolgsversprechendsten betrachtet, da sie mit höherer Wahrscheinlichkeit schon bei verhältnismäßig niedrigen Temperaturen abläuft. Bisher gibt es allerdings noch kein Kraftwerk, das Fusion als wirtschaftliche Energiequelle nutzt. Es wird jedoch an verschiedenen Reaktorkonzepten geforscht, um ein solches zu realisieren.\\

Eines der größten Probleme beim Bau eines Fusionsreaktors ist der Einschluss des Plasmas, in dem die Fusionsreaktionen ablaufen. Der Grund dafür liegt in den hohen Plasmatemperaturen von mehreren \num{100}~Mio.~\unit{\degreeCelsius} und damit einhergehenden thermischen Belastungen, denen kein Material standhält. Auf Dauer akzeptabel sind nur \qty[per-mode=symbol]{10}{\MW\per\m\tothe{2}}, was bei direktem Kontakt zwischen Material und Plasma überschritten wird. Demnach muss der Plasmaeinschluss anderweitig gestaltet werden. Die vielversprechendsten Ansätze sind der Inertialeinschluss und der magnetische Einschluss. Bei letzterem schließt ein toroidal geschlossenes und poloidal verdrilltes Magnetfeld das Plasma ein. Zur Realisierung eines solchen Magnetfelds gibt es wieder zwei Hauptkonzepte: Tokamaks und Stellaratoren. Auf das Prinzip eines Stellarators wird an dieser Stelle genauer eingegangen.\\

Stellaratoren kennzeichnen sich durch die Eigenschaft, dass ihr Magnetfeld ausschließlich durch externe Spulen erzeugt wird. Das bedeutet, es ist kein Plasmastrom notwendig, um das Magnetfeld aufzubauen. Dieser Umstand bringt den Vorteil mit sich, dass kontinuierlicher Betrieb möglich ist und Instabilitäten durch Änderung des Plasmastroms geringer ausfallen. Der Preis dafür ist eine komplexe Geometrie der Spulen und der Wegfall vollständiger toroidaler Symmetrie. Stattdessen tritt modulare Symmetrie in Reaktor und Magnetfeld auf.\\

Der Stellarator Wendelstein 7-X (W7-X) des Max-Planck-Instituts für Plasmaphysik in Greifswald ist eine der größten und am weitesten entwickelten Anlagen dieser Art. Er ist seit 2015 in Betrieb und untersucht vornehmlich Wasserstoffplasmen (H\({^+}\)). Die Anlage sowie das Magnetfeld weisen eine fünfzählige toroidale Symmetrie auf, die sich in der Unterteilung in fünf Module (M1-M5) wiederspiegelt wie Abb. \ref{fig:DivertorTorus} zeigt. Außerdem besteht jedes Modul aus zwei identischen, \qty{180}{\degree}-rotationssymmetrischen Halbmodulen (HMs) - einem oberen und einem unterem. Auch diese Symmetrie teilen Anlage und Magnetfeld.\\

\begin{figure}[!htb]
	\centering
	\includegraphics[scale=0.8]{figures/DivertorTorus.png}
	\caption{Plasmakontur der Standardkonfiguration EIM (orange) und Position der Divertoreinheiten (schwarz) in den fünf Modulen M1-M5 von Wendelstein 7-X. Die Abbildung wurde in \cite{ArunLP} veröffentlicht}.\label{fig:DivertorTorus}
\end{figure}

Der allgemeinen Aufbau des durch das Magnetfeld eingeschlossenen Plasmas ist der Folgende: Im Inneren liegt das besonders heiße Core Plasma und außen herum sind die kühleren magnetischen Inseln angeordnet, welche das Randplasma bilden. Getrennt sind diese Bereiche durch die sogenannte Separatrix, dargestellt wird diese Struktur in Abb. \ref{fig:PlasmaKontur}. Während das Core Plasma geschlossene Magnetfeldlinien aufweist, werden die der magnetischen Inseln von besonders widerstandsfähigen Wandkomponenten des Plasmagefäßes - den sogenannten Divertoren - geschnitten. Die verschiedenen Wandkomponenten werden in Abb. \ref{fig:PFC} veranschaulicht.\\

Durch das Anlegen verschiedener Spulenströme kann das Magnetfeld in einer Vielzahl von Konfigurationen gefahren werden, die sich in der Form des eingeschlossenen Plasmas unterscheiden. Die am häufigsten vertretenen Magnetfeldkonfigurationen sind die Low-Iota Konfiguration DBM, die Standardkonfiguration EIM, die High-Mirror Konfiguration KJM und die High-Iota Konfiguration FTM. Iota ist dabei die Kenngröße der Rotationstransformation und drückt die Verdrillung des Magnetfelds aus. Für jede Magnetfeldkonfiguration variiert die Position der magnetischen Inseln und damit auch die Position, an der die Wandkomponenten ihre Magnetfeldlinien schneiden. Die dadurch entstehende in Abb. \ref{fig:wettedArea} gezeigte Kontaktfläche wird Strikeline genannt und kennzeichnet sich durch höhere Teilchenflussdichten der auftreffenden Teilchen. Dadurch wird diese Region stärker erhitzt und muss entsprechend widerstandsfähig sein. Aus diesem Grund liegt die Strikeline auf dem High Heat Flux Carbon Fibre Composite Divertor, einem wassergekühlten Wandelement, das solchen thermischen Belastungen standhält.\\
\begin{figure}[!htb]
	\centering
	\includegraphics[scale=0.55]{figures/PlasmaKontur.png}
	\caption{Form des Plasmas in der Standardkonfiguration EIM mit Core Plasma (schwarz) und magnetischen Inseln (farbig). Oben ist die Plasmakontur des gesamten Torus dargestellt, unten die zugehörigen Poincare Plots. Diese zeigen die geschlossenen magnetischen Flussflächen in Querschnitten des Plasmas bei unterschiedlichen toroidalen Winkeln. Die Position der Winkel ist Abb. \ref{fig:DivertorTorus} zu entnehmen. Die Abbildung wurde in \cite{tempLimit2} veröffentlicht.}\label{fig:PlasmaKontur}
\end{figure}

\begin{figure}[!htb]
	\centering
	\includegraphics[scale=0.6]{figures/PFC.png}
	\caption{In-vessel view einer Divertoreinheit mit den einzelnen Wandkomponenten und den maximal zulässigen Temperaturen. Mittig dargestellt ist der Divertor, welcher die höchste thermische Belastung aushält. Die Abbildung wurde in \cite{tempLimit2} veröffentlicht.}\label{fig:PFC}
\end{figure}

\begin{figure}[!htb]
	\centering
	\includegraphics[scale=0.55]{figures/wettedArea.png}
	\caption{Position der Strikeline (rot) auf dem Divertor für verschiedene Magnetfeldkonfigurationen. Von links nach rechts: Low-Iota Konfiguration DBM, Standardkonfiguration EIM, High-Mirror Konfiguration KJM und High-Iota Konfiguration FTM. Diese Abbildung wurde aus \cite{MfieldsAndWettedArea} entnommen.}\label{fig:wettedArea}
\end{figure}

Pro Halbmodul gibt es eine Divertoreinheit, wie in Abb. \ref{fig:DivertorTorus} zu sehen ist. Der Aufbau einer solchen Einheit ist in Abb. \ref{fig:DU} dargestellt und im Folgenden beschrieben. Es wird unterteilt in vertikale und horizontale Targets, welche durch den Pumpspalt getrennt sind. Weiterhin kann nach Targetmodulen (TMs) und innerhalb dieser in Targetelemente (TEs) unterschieden werden. Zusätzlich können noch drei Bereiche des Divertors festgelegt werden: Der low-iota Bereich, der zentrale Teil und die high-iota Region. Diese Bezeichnungen rühren von der Position der Strikeline für Magnetfeldkonfigurationen mit unterschiedlichen Iota-Werten her. Starke Magnetfeldverdrillung (hohes Iota) wie bei FTM erzeugt eine Strikeline im high-iota Bereich des Divertors, schwache Verdrillung (niedriges Iota) wie bei DBM, EIM und KJM eine Strikeline im low-iota Bereich.\\

\begin{figure}[!htb]
	\centering
	\includegraphics[scale=0.5]{figures/LP_position.png}
	\caption{CAD Modell einer Divertoreinheit bestehend aus den durch den Pumpspalt getrennten vertikalen und horizontalen Targets. Weiterhin ist die Unterteilung in Targetmodule (TMs), bestehend wiederum aus Targetelementen möglich. Die drei Bereich low-iota, central part und high-iota stellen ebenfalls eine mögliche Einteilung dar. Rot hervorgehoben sind die Targetelemente TM2h07 und TM3h01 im low-iota Bereich sowie TM8h01 im high-iota Bereich. Auf ihnen befinden sich die Langmuir Sonden. Diese Grafik basiert auf einer in \cite{tempLimit2} veröffentlichten Darstellung.}\label{fig:DU}
\end{figure}

Um das Verhalten des Plasmas während einer einzelnen Entladung beobachten zu können, ist eine Vielzahl unterschiedlicher Diagnostiken in W7-X verbaut, die die Bestimmung verschiedener Plasmaparameter erlaubt. Zu diesen Diagnostiken zählen unter anderem die im low-iota und high-iota Bereich des horizontalen Divertors angebrachten Langmuir Sonden (LS). Sie ermöglichen die Bestimmung der Elektronendichte und -temperatur im Randplasma durch die Messung der Stromstärke in Abhängigkeit der an die Sonde angelegten Spannung.\\

Für zwei der 36 verbauten Langmuir Sonden ist bekannt, das in OP2.2 und OP2.3 - den neusten beiden Experimentalkampagnen - bei einigen Messungen Fehler aufgetreten sind. Elektronendichte und -temperatur erreichten unrealistische Werte, die durch einen Abfall des Widerstands zwischen Sonde und Ground verursacht wurden. Welche Messungen betroffenen sind und warum dieses Phänomens aufgetreten ist, ist nicht endgültig geklärt.\\

Diese Arbeit widmet sich der Untersuchung der durch Widerstandsabfälle verursachten Fehlmessungen. Eine systematische Analyse aller Messungen der beiden betroffenen Langmuir Sonden mithilfe einer neu entwickelten Programmroutine zur Detektion dieses Phänomens ermöglicht die Unterscheidung der Messwerte in zuverlässige und unzuverlässige Messdaten. Eine Erfassung der Betriebs- und Plasmaparameter zum Zeitpunkt der Fehlmessungen erlaubt außerdem Rückschlüsse auf eine potentiell dahinterstehende Systematik. Daraus können mögliche Ursachen für das Auftreten der Widerstandsabfälle abgeleitet werden. Unter Umständen ist dadurch die Behebung des Problems möglich.\\

In Kapitel \ref{chap:theory} werden die Langmuir Sonden und ihre Funktionsweise allgemein und W-7X spezifisch vorgestellt. Kapitel \ref{chap:calculations} beschreibt den Detektionsprozess für Fehlmessungen sowie die Datenerfassung der Betriebs- und Plasmaparameter. Anschließend werden die Ergebnisse der Analyse in Kapitel \ref{chap:results} präsentiert und in Kapitel \ref{chap:discussion} diskutiert. Eine kurze Zusammenfassung der Arbeit wird in Kapitel \ref{chap:summary} gegeben, während in Kapitel \ref{chap:outlook} Erweiterungen des Projekts sowie Konsequenzen der gefundenen Ergebnisse auf den Betrieb der Langmuir Sonden in zukünftigen Experimentalkampagnen erläutert werden. Die Kapitel \ref{chap:appendixA} und \ref{chap:appendixAa} beinhalten die Erklärung der Formelzeichen und Abkürzungen. 