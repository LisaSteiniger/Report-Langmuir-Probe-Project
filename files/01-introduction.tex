\chapter{Einleitung} \label{chap:1}
Fusion stellt eine Möglichkeit der sauberen Energiegewinnung dar, die durch den weltweit steigenden Energiebedarf und die Bedrohung durch den Klimawandel immer mehr Interesse auf sich zieht. Dabei wird die Eigenschaft von Fusionsreaktionen - also dem Verschmelzen zweier Atomkerne zu einem schwereren Atomkern sowie unter Umständen einem subatomaren Teilchen - genutzt, dass ein Teil der Masse der Ausgangsstoffe in Energie umgewandelt und abgestrahlt wird. In der Regel wird die Reaktion der Wasserstoffisotope Deuterium und Tritium als am erfolgsversprechendsten betrachtet, da sie mit höherer Wahrscheinlichkeit schon bei verhältnismäßig niedrigen Temperaturen abläuft. Bisher gibt es allerdings noch kein Kraftwerk, das Fusion als wirtschaftliche Energiequelle nutzt. Es wird jedoch an verschiedenen Reaktorkonzepten geforscht, um ein solches zu realisieren.\\

Eines der größten Probleme beim Bau eines Fusionsreaktors ist der Einschluss des Plasmas, in dem die Fusionsreaktionen ablaufen. Der Grund dafür liegt in den hohen Plasmatemperaturen von mehreren \num{100}~Mio.~\unit{\degreeCelsius} und damit einhergehenden thermischen Belastungen, denen kein Material standhält. Auf Dauer akzeptabel sind nur \qty[per-mode=symbol]{10}{\MW\per\m\tothe{2}}, was bei direktem Kontakt zwischen Material und Plasma überschritten wird. Demnach muss der Plasmaeinschluss anderweitig gestaltet werden. Die vielversprechendsten Ansätze sind Inertialeinschluss und magnetischer Einschluss. Bei letzterem schließt ein toroidal geschlossenes und poloidal verdrilltes Magnetfeld das Plasma ein. Zur Realisierung eines solchen Magnetfelds gibt es wieder zwei Hauptkonzept - Tokamaks und Stellaratoren. Auf das Prinzip eines Stellarators wird an dieser Stelle genauer eingegangen.\\

Stellaratoren kennzeichnen sich durch die Eigenschaft, dass ihr Magnetfeld ausschließlich durch externe Spulen erzeugt wird. Das bedeutet, es ist kein Plasmastrom notwendig, um das Magnetfeld aufzubauen. Dieser Umstand bringt den Vorteil mit sich, dass kontinuierlicher Betrieb möglich ist und Instabilitäten durch Änderung des Plasmastroms geringer ausfallen. Der Preis dafür ist eine komplexe Geometrie der Spulen und der Wegfall vollständiger toroidaler Symmetrie. Stattdessen tritt modulare Symmetrie in Reaktor und Magnetfeld auf.\\

Der Stellarator Wendelstein 7-X (W7-X) des Max-Planck-Instituts für Plasmaphysik in Greifswald ist eine der größten und am weitesten entwickelten Anlagen dieser Art. Er ist seit 2015 in Betrieb und untersucht vornehmlich Wasserstoffplasmen (H\({^+}\)). Die Anlage sowie das Magnetfeld weisen eine fünfzählige toroidale Symmetrie auf, die sich in der Unterteilung in fünf Module (M1-M5) wiederspiegelt. Außerdem besteht jedes Modul aus zwei identischen, \qty{180}{\degree}-rotationssymmetrischen Halbmodulen (HM) - einem oberen und einem unterem.\\

Durch das Anlegen verschiedener Spulenströme kann eine Vielzahl Magnetfeldkonfiguration gefahren werden, die sich in der Form des eingeschlossene Plasma unterscheiden. Sie alle teilen sich jedoch den allgemeinen Plasmaaufbau aus dem im Inneren liegenden besonders heißen Core Plasma und den außen herum angeordneten kühleren magnetischen Inseln, welche das Randplasma bilden. Getrennt sind diese Bereiche durch die sogenannte Separatrix. Während das Core Plasma geschlossene Magnetfeldlinien aufweist, werden die der magnetischen Inseln von Wandkomponenten des Plasmagefäßes geschnitten.\\

Die am häufigsten vertretenen Magnetfeldkonfigurationen sind die Low-Iota Konfiguration DBM, die Standardkonfiguration EIM, die High-Mirror Konfiguration KJM und die High-Iota Konfiguration FTM. Iota ist dabei die Kenngröße der Rotationstransformation und drückt die Verdrillung des Magnetfelds aus. Für jede Magnetfeldkonfiguration variiert die Position der magnetischen Inseln und damit auch die Position, an der die Wandkomponenten ihre Magnetfeldlinien schneiden. Die dadurch entstehende Kontaktfläche wird Strikeline genannt und kennzeichnet sich durch höhere Teilchenflussdichten der auftreffenden Teilchen. Dadurch wird diese Region stärker erhitzt und muss entsprechend widerstandsfähig sein. Aus diesem Grund liegt die Strikeline auf dem High Heat Flux Carbon Fibre Composite Divertor, einem wassergekühlten Wandelement, das solchen  thermischen Belastungen standhält.\\

Pro Halbmodul gibt es eine Divertoreinheit. Diese wird unterteilt in vertikale und horizontale Targets, welche durch den Pumpspalt getrennt sind. Weiterhin kann nach Targetmodulen (TMs) und innerhalb dieser in Targetelemente (TEs) unterschieden werden. Zusätzlich können noch drei Bereiche des Divertors festgelegt werden: Der low-iota Bereich, der zentrale Teil und die high-iota Region. Diese Bezeichnungen rühren von der Position der Strikeline für Magnetfeldkonfigurationen mit unterschiedlichen Iota-Werten her. Starke Magnetfeldverdrillung (hohes Iota) wie bei FTM erzeugt eine Strikeline im high-iota Bereich des Divertors, schwache Verdrillung (niedriges Iota) wie bei DBM, EIM und KJM eine Strikeline im low-iota Bereich.\\

%%
